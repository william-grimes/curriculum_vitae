\subsection{DareData Engineering \hfill 2021 -- 2023}
    \begin{list2}
        \item Led a team as the machine learning engineer on a project that improved call center operations by developing and implementing a client-operator matching system. This system increased profitability, customer satisfaction, and reduced customer churn.
        \item Used PySpark MLlib to successfully train a Gradient Boosted Tree (GBT) regressor that ranked clients with the best operator from approximately 500 million operator-customer pairs. Delivered daily leads that were operationalized in the call center, leading to improved efficiency and productivity.
    \end{list2}

\subsection{Numerico Technologies \hfill 2018 -- 2021}
    \begin{list2}
         \item As a machine learning engineer I played a pivotal role in the development of a product that significantly reduces response time for inspectors attending incidents on Dutch highways. I conducted extensive modelling and research, and the resulting Monte-Carlo method approach effectively incorporates real-time information on locations, traffic conditions, and incidents to optimally assign inspectors to road sections with the highest likelihood of incident. During the pilot phase, the software delivered an impressive average 20\% reduction in response time, with the greatest impact observed during low-coverage periods. Subsequently, the product was successfully rolled out across all regions in the Netherlands, resulting in remarkable improvements in incident response times.
         \item Maintained and designed a robust project infrastructure ensuring seamless deployments across development, staging, and production servers. This leveraged tools such as Jenkins for deployment, Kafka and Redis for coordinating data flows between microservices, Docker for containerisation, PostgreSQL for data persistence. The resulting infrastructure was highly resilient and contributed significantly to the success of the project.
         \item Led data science research projects on the use of camera data with Yolo v4 to detect vehicles and monitor traffic flow and speed, and analysis of accident blackspots on highways.
    \end{list2}

\subsection{DSSG (Data Science for Social Good) \hfill 2017 -- 2018}
    \begin{list2}
        \item Developed an open-source fishing risk web application using RShiny, combining vessel tracking data with satellite imagery, scoring vessels according to likelihood of illegal fishing behaviours.
        \item Created a model for predicting road accidents in the Netherlands using a bouquet of classifiers (XGBoost, random forest, SVM) with road features, weather features, and temporal features.
        %\item Learning car accidents occurrence probability on the Dutch highways using transport, weather and road data.
        % \item Established AWS EC2 and PostgreSQL infrastructure for several teams, whilst mentoring a team of four, giving them technical advice and performing code reviews.
    \end{list2}

\subsection{Faculty AI \hfill 2017}
    \begin{list2}
        %\item Fellowship programme in commercial data science for researchers with strong analytic background.
        \item Implemented topic modelling on a corpus of 37,000 fire incident reports from the London Fire Brigade , revealing previously unknown fire scenarios.
        % \item Finished intensive eight week training in machine learning, databases, and statistics.
    \end{list2}

    %\textbf{\listing Laboratory for Molecular Cell Biology (LMCB)} \vspace{2mm}\\\vspace{1mm}%
    %\textsl{PhD Student} \hfill \textbf{Sep 2013 -- Nov 2016}
    %\begin{list2}
    %    \item Joint scholarship at LMCB and A*STAR Bioinformatics Institute, Singapore.
    %	\item Developed automated image analysis pipelines for use in high-throughput microscopy assays, using machine learning to classify cell organelles.
    %	%\item Published articles in \emph{Nature Scientific Reports} and \emph{Journal of Thrombosis and Haemostasis}.
    %\end{list2}

\subsection{National Institute of Informatics \hfill 2013}
    \begin{list2}
      \item Wrote software in Java for annotation and classification of phenotypes in $\mu$CT images of mice.
    \end{list2}
